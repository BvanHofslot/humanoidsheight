%
\chapter{Conclusion}%
\label{chap:conclusion}
There are numerous ways to describe the dynamical behavior of a walking robot, that are often sufficient for a constant gait over flat terrain. Those assume most of the times a \ac{LIP} model. If there is looked at examples of human behavior by maneuvering over un-flat terrain or during rapid changes in velocity, the constant height assumption might not be always sufficient. With this survey there is an insight gained in the most common robot modeling, planning and control strategies and observed that the height variation of the robot is often not captured in its dynamical description. A selection of one of the most relevant publications concerning this research focus was reviewed, but hardware results are often not shown at robots similar to that in use at IHMC: Atlas or Valkyrie. Furthermore, oftentimes assumptions and constraints were applied to equations, which not have shown yet to give better results than existing models. It seems like an interesting challenge to investigate the effects of height variation in the robot model and to see to what extend this can be applied on a physical robot. The final implementation may have the more complicated shape of a trajectory planning or a \ac{MPC} problem, or may use simple heuristics that are implemented in a smart way. In a control problem computation time is an important consideration, and different strategies have to be judged on speed as well.