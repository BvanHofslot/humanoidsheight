%
\chapter{Conclusion}%
\label{chap:conclusion}
There are numerous ways to describe the dynamical behavior of a walking robot, that are often sufficient for constant walking gaits over flat terrain that are not very dynamic. Those descriptions make most of the times use of a \ac{LIP} model. If there is looked at examples of human behavior by maneuvering over un-flat terrain or during rapid changes in velocity, the constant height assumption is not always sufficient. \\
With this survey there is an insight gained in the most common humanoid robot modeling, planning and control strategies and observed that \ac{CoM} height variation of the robot is often not captured in its dynamical description. A selection of one of the most relevant publications concerning the research of focus was reviewed, but hardware results are often not shown at robots similar to that in use at IHMC: Atlas or Valkyrie. Furthermore, oftentimes constraints were applied to system descriptions for planning and control, which not have shown yet to give better results than existing models.\\
The discussed methods that link to \ac{CoM} height variations can be broken down based on:
\begin{itemize}
\item \ac{2D} and \ac{3D} methods.
\item Methods that need a predefined footstep plan and timing and methods that capture this in the problem.
\item Computation times.
\item Theory or application orientated methods.
\item Robustness against disturbances, in the case a control method.
\end{itemize}
Concerning the goal of application, \ac{2D} side-view methods, like \cite{pratt2007derivation, koolen2016balance, gao2017increase, nguyen2017dynamic}, need special care compared to \ac{3D} methods, like \cite{englsberger2013three, hopkins2014humanoid, liu2015trajectory, caron2018capturability}. Methods that need a predefined footstep plan and timing, like  \cite{englsberger2013three, hopkins2014humanoid,  caron2018capturability} have an advantage above those who do not. Strategies with faster computation times, like the analytic \ac{MPC} approach in \cite{koolen2016balance}, are more suited for real-time usage. Robustness properties are an important factor in control, but are often not shown.\\
Regarding the exploration of the effects of height variation, all different insights have a contribution to this. Fundamental theory can be explanatory and might be useful to extend for application. \\
This literature survey emphasizes the idea that it is an interesting challenge to investigate the effects of height variation in the humanoid robot model and to see to what extend this can be applied on a physical robot. The goal in the research project will be to develop new varying height models and to build on existing models to extend current knowledge and to find an application on hardware to show the usefulness.\\
The final implementation may have the more complicated shape of a trajectory planning or a \ac{MPC} problem, or may use simple heuristics that are implemented in a smart way. In a control problem computation time is an important consideration, and different strategies have to be judged on speed as well. 