%
\chapter{Varying Height Modeling, Planning \& Control}%
\section{Varying CoM Height Planning and Control}
As mentioned earlier, the recently work has been done in keeping the constant height assumption away from the walking model. The motivation for this is twofold. The use of height can be used in tracking control. Where angular momentum and \ac{CoP} are the two control inputs for a \ac{LIP} model, the use of height, thus a varying leg force, can be used as a third input. Furthermore, in motion over un-flat terrain a varying height model can give a better approximation of how the dynamics will behave over time. In other word: planning can be improved. 
\subsection{\ac{DCM} for varying height}
One attempt to use a varying height model for planning is the introduction of the \ac{DCM} in \cite{englsberger2013three}. This is used in a rough terrain planning model. The \ac{DCM} is defined as
\begin{equation}
\boldsymbol{\xi} = \boldsymbol{x} + b\boldsymbol{\dot{x}}
\end{equation}
where $\boldsymbol{\xi}=[\xi_x,\xi_y,\xi_z]^T$ is the \ac{DCM}, $\boldsymbol{x}=[x_{CoM}, y_{CoM}, z_{CoM}]^T$ and $\boldsymbol{\dot{x}}=[\dot{x}_{CoM}, \dot{y}_{CoM}, \dot{z}_{CoM}]^T$ are the \ac{CoM} position and velocity and $b>0$ is the time-constant of the \ac{DCM} dynamics. 
[ZOEK UIT HOE VRP EN ECMP VERKREGEN ZIJN].
\cite{hopkins2014humanoid} defines the constant $b=1/\omega_0$, where $\omega_0=\sqrt{\frac{g}{z_{CoM}}}$. This makes it the same as the \ac{ICP}, but with a $z$ component in the state. The \ac{DCM} is derived with respect to time, where $\omega$ is posed to be time varying, so the height is time varying. This brings a new expression for the so called \ac{DCM}.
\cite{caron2018capturability} uses the \ac{DCM} for a MPC control scheme to capture the inverted pendulum \ac{CoM}.

\subsection{Orbital Energy for nonlinear \ac{CoM} trajectories}
The true non-linear model of a traveling \ac{CoM} subject to a force coming from the $[0,0]$ point on the ground is for the \ac{2D} $xz$\nobreakdash-plane
\begin{eqnarray}
m\ddot{x} = F \frac{x}{\sqrt{x^2+z^2}}\\
m\ddot{z} = -mg+F \frac{z}{\sqrt{x^2+z^2}}
\label{eq:nonlindyn}
\end{eqnarray} 
where $F$ is the ground reaction force. \cite{pratt2007derivation} constrains $z=f(x)$ and integrates  the equations of motion over time, how a similar conservation of Orbital Energy as in Eq. \eqref{eq:Elip} is derived. This Orbital Energy however is not based on a straight line, but depends on the function $f(x)$. The  full integration can be found in the paper, but the final expression writes as
\begin{equation}
    \frac{1}{2}\dot{x}^2\bar{f}^2(x)+gx^2f(x) - 3g\int_{x_0}^xf(\xi)\xi d\xi = \frac{1}{2}\dot{x}_0^2\bar{f}^2(x_0)+gx_0^2f(x_0).
\end{equation}
where $\bar{f}(x)=f(x)-f'(x)x$, $[x_0,\dot{x}_0]$ is the initial horizontal state and velocity and $[x,\dot{x}]$ is the current horizontal state and velocity. There is experimented in simulation with a symmetric polynomial, where the trajectory is tracked using a PD feedback scheme where the leg force is the control input. The polynomial description makes the integral directly solvable. To make a clear seperation between \ac{Elip}, this energy form will be defined later as \ac{Eorbit}.\\
Recently, this is used in a \ac{MPC} formulation and basic simulations have been done by \cite{koolen2016balance}. The main idea in this paper is to define four constraints on the trajectory $f(x)$, which define the four constants of a cubic polynomial by a matrix inversion. Those consist of two constraints on the initial conditions, one constraint on the final condition and one constraint on the Orbital Energy conservation. The matrix inversion is done offline, so that the control input can directly be written as a function of the polynomial constants and thus of the initial and final states. 

\subsection{Other varying height approaches}
Besides derivations of the \ac{ICP} for varying height and the \ac{Eorbit}, there are other approaches that describe the effects of varying height. The authors of \cite{gao2017increase} come up with different strategies to use varying height to generate impact or to relatively speed up the \ac{CoM} motion compared to the \ac{ICP} trajectory. Besides gravity, an extra vertical acceleration constant is added in the \ac{LIP} equations of motion. \\
An applied approach is described in \cite{nguyen2017dynamic}. For different step lengths and step heights, trajectories are generated offline using a direct collocation optimization framework. Online is interpolated between trajectories with knowledge of the current step and the upcoming step. The trajectories are on joint level and make use of the earlier discussed \ac{HZD} approach. This is applied on hardware, but with the sagittal motion supported and thus only the \ac{2D} dynamics in the $xz$-plane are considered.

\subsection{Trajectory optimization for nonlinear systems}
As becomes clear, looking at the simplified varying height robot model, nonlinearity is the main bottleneck. As such, it would be rewarding to have already an insight in trajectory optimization for such systems. \cite{kelly2017introduction} poses a MATLAB  toolbox that can be used to solve trajectory optimization problems. Furthermore, it gives a lot of information how the solvers work. The methods distinguish themselves between direct and indirect methods, shooting and collocation methods and the difference between the so called \'h'- and \'p'-methods. The latter two define the degree of segmentation and the order of used functions between the segments of the problem. Direct collocation methods are most likely best suited for the subject of interest, as they require less computation time than indirect and shooting methods. Using the system description of Eq. \eqref{eq:nonlindyn}, a direct collocation method could be used for generating a capture trajectory for example. 
