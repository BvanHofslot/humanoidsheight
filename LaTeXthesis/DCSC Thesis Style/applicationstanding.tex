\chapter{Toward Application: Standing}\label{chap:standing}

\section{Experimental Setup}
Applying a lower weight in the whole-body \ac{QP} on vertical linear momentum rate, causes the robot to use height variation in recovery.

\begin{table}[ht]
\caption{Major affected tasks by $\mathbf{\dot{l}}_{xy,d}$ in whole-body QP, if \ac{CoP} is saturated.} % title of Table
\centering % used for centering table
\begin{tabular}{c c c } % centered columns (4 columns)
\hline\hline %inserts double horizontal lines
Affected Desired & Constraint/Consideration & Centroidal Momentum Rate \\
%heading
\hline % inserts single horizontal line
 $z_d$ & Leg singularity & Linear\\
 $\boldsymbol{\phi}_{body,d}$ & Upper body pose & Angular\\
 $\mathbf{r}_{foot,d}$ &  Touchdown time & Angular\\
%[1ex] % [1ex] adds vertical space
\hline %inserts single line
\end{tabular}
\label{tab:eatqp} % is used to refer this table in the text
\end{table}

\section{Method}
Other publications that consider \ac{CoM} height variations for balance use \ac{MPC}. Considering worst-case scenario's, where additional horizontal force is needed, `the best you can do' is needed to not fall over. This motivates to use a proportional controller in worst case scenario's, next to the benefit of simplicity and robustness.

%Challenges
\subsection{Motivation}
The controller that uses \ac{CoM} height variations for balance is developped under the motivation that the robot should do the best it can to not fall over. Height variation has the most effect on the horizontal momentum if the robot is relatively far from the point foot \todo{ref section}. Therefore, situations where the controller is allowed to work considers only the scenario's where the \ac{CoP} is saturated.
\paraskip
To make a good comparison with other strategies possible, the controller is designed with the objective to not request additional angular momentum from the robot. Other considerations are kinematic reachability, control in \ac{3D} and the predictability of the dynamics of the system.

%Architecture
\subsection{Control Framework}
To actively change $\dotldxy$, the objective is to generate as few resulting additional angular momentum as possible, compared to the normal control setup. Note that the normal control setup relies on a computed $\rcmpd$ based on the linear model with constant height, as in Equation \eqref{eq:rcmpd} and \eqref{eq:dotldxy}. In the case of the assumption of a \ac{LIP}, the expected angular momentum generated by the robot is a linear function depending on the difference between the \ac{CMP} and \ac{CoP}. If height changes are considered however, having the \ac{CMP} in the same location in one configuration, can request different angular momenta with changing $\fgrz$. The generic formula to write resulting horizontal ground reaction force as a function of the \ac{CMP}  location is:
\begin{equation}
\fgrxy=\frac{\mathbf{c}_{xy}-\mathbf{r}_{cmp}}{z}\fgrz
\label{eq:fgrxygeneric}
\end{equation}
If $\fgrz$ varies and $\rcmp$ remains constant, $\taucom$ has to scale linearly with $\fgrz$, because $\rcmp = \rcop + \frac{\taucom}{\fgrz}$. Therefore, additional horizontal linear momentum rate is requested based on the \ac{CoP}, rather than the \ac{CMP}.
\paragraph{Generation of $\dotldxy$} is done by modifying the desired linear momentum rate from the \ac{LIP}-based PD-controlled setup. Equation \eqref{eq:fgrxygeneric} is rewritten in terms of $\rcop$ in the following way:
\begin{align}
\fgrxy&=\frac{\mathbf{c}_{xy}-(\mathbf{r}_{cop}+\frac{\taucom}{F_z})}{z}\fgrz \\
&=\frac{\mathbf{c}_{xy}-\mathbf{r}_{cop}}{z}m(g+\ddot{z}) - \frac{\taucom}{z} \\
&=\frac{\mathbf{c}_{xy}-\mathbf{r}_{cop}}{z}mg - \frac{\taucom}{z} + \frac{\mathbf{c}_{xy}-\mathbf{r}_{cop}}{z}m\ddot{z}.
\end{align}
Note that for a \ac{LIP} model the following equality holds:
\begin{equation}
	\frac{\mathbf{c}_{xy}-\mathbf{r}_{cop}}{z_0}mg - \frac{\taucom}{z_0} = \frac{\mathbf{c}_{xy}-\mathbf{r}_{cmp}}{z_0}mg.
\end{equation}
For this reason the desired linear momentum rate is modified in the following way:
 \begin{equation}
\dot{\mathbf{l}}_{xy,d}=\underbrace{ \frac{\mathbf{c}_{xy}-\mathbf{r}_{cmp,d}} {z_0}mg}_{\dotldxylip}  + \underbrace{\frac{\mathbf{c}_{xy}-\mathbf{r}_{cop,d}}{z}m\ddot{z}_d}_{\dotldxyheight},
\end{equation}
where $\dotldxylip$ is the already existing desired horizontal linear momentum rate of change from the \ac{ICP} controller and $\dotldxyheight$ is the additional desired momentum rate from height control. The desired \ac{CoP} $\rcopd$ is equal to $\rcmpd$ if $\rcmpd$ is inside the convex support polygon. If $\rcmpd$ is outside the polygon, $\rcopd$ is the orthogonal projection of $\rcmpd$ onto the polygon, as the \ac{CoP} lives by definition inside the support polygon. The desired height acceleration $\ddzd$ is the output of the height controller, which will be discussed in Section \ref{sec:strategy}.
\paragraph{The condition to activate the controller} is when the \ac{ICP} error $\icpe$ is of such length, that \ac{CoP} strategies are no longer sufficient for balance control and an additional input has to be generated. The condition is $\icpe>\fracicp$, where the parameter $\fracicp=0.04$ is found to be a good estimate of the moment when the \ac{CoP} hits the edge of the polygon, considering the control settings at the time of writing \todo{ref to settings}. Also, the controller is only allowed to work in a static case or during the swing phase.

\paragraph{Bang-bang control}  is initially assumed to be used in the controller. As the robotic system has limits on kinematics, joint-velocities and joint torques, the possible upward acceleration is limited. Downwards, the system might suffer from contact friction problems if the system drops with high acceleration. These limitations on upward and downward acceleration are approximated with a constant maximum value, which will be used in a worst-case scenario, where the robot is about to tip and the system needs to do the best it can to avoid falling.

\section{Results}
Which cases?
\begin{itemize}
	\item Normal limit v.s. height control on normal limit
	\item Normal limit v.s. height control limit -> capture limits
	\item Angular momentum v.s. height control -> \todo{How?}
\end{itemize}
\subsubsection{Simulation}
\begin{itemize}
	\item 360 push limits
	\item Front and side push deep evaluation.
\end{itemize}
\subsubsection{Hardware}
\begin{itemize}
	\item Increase $\ddzdmax$ until best found
	\item Front and side push
	\item Compare with normal config
	\item Compare with QP-based
	\item Compare with ang momentum
\end{itemize}
On what?
\begin{itemize}
	\item joint torques: ankle, knee, hip, back
	\item pose: angular momentum rate, angular momentum, body pose error, CoM
	\item reference points: CMP, CoP, ICP -> Integrated CoP effect and Angular effect > linear momentum
\end{itemize}

\section{Discussion}