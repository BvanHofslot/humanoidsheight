%
% Abstract (does not appear in the Table of Contents)
\chapter*{Abstract}%

Traditional balance strategies for humans and humanoid robots are changing the \ac{CoP} position, a result of the `ankle strategy', change of body angular momentum, for example by a `hip strategy', and taking a step. For humanoid robots, a common assumption behind these strategies is that the \ac{CoM} height is constant, or at least predefined. However, height changes can be used as an input for balance control, as for example can be observed by an athlete performing a long jump. In this work, \ac{CoM} height variations are considered as an input for balance control. The results are compared with constant height approaches.

The first contribution proposes bounds on the initial states of the \ac{VHIP} from which convergence is possible, also known as capture regions. First, only a unilateral contact constraint is considered; negative \ac{CoM} acceleration cannot be smaller than gravitational acceleration. Second, \ac{CoM} height constraints are added to the model, after which a capture region can still be computed closed-form. Third, vertical acceleration constraints are added, after which capture regions are computed numerically using a bang-bang control law. The last capture region bridges the transition to the applied part of this work.

The second contribution proposes a bang-bang control law, suitable for application on humanoid robot using momentum-based control. Push recovery is tested on NASA's Valkyrie humanoid robot while the robot is standing. In simulation, an increase in recoverable push of $9$\% can be observed compared to a controller that only uses \ac{CoP}, when pushing the back of the robot. On hardware, an average increase of $7$\% can be observed for this push direction using a load sensor. Additionally, tests are conducted on hardware on Boston Dynamics' Atlas using a medicine ball on a rope, but no improvement in recovery was observed. The control method for standing push recovery is also extended for the use in walking scenarios. For Valkyrie in simulation, recovery improved the most for a push applied in the first phase of the single support state for rear and frontal push directions. 