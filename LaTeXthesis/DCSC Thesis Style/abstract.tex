%
% Abstract (does not appear in the Table of Contents)
\chapter*{Abstract}%

This research considers using \ac{CoM} height variation as an input for balance control on a humanoid robot. Traditional balance strategies for humanoid robots are taking a step, control of the \ac{CoP} location, a result of the `ankle strategy', and changing the angular momentum about the \ac{CoM}, for example by a `hip strategy'. For humanoid robots, a common assumption behind these strategies is that the \ac{CoM} height is predefined. However, \ac{CoM} height changes can be used as an input for balance control, as for example can be observed during the landing of an athlete after a long jump. 

The first contributions of this work are bounds on the initial states for the \ac{VHIP} from which convergence is possible, also known as capture regions. First, only a unilateral contact constraint is considered; negative \ac{CoM} acceleration cannot be smaller than gravitational acceleration. Second, \ac{CoM} height constraints are added to the model, after which a capture region can still be computed in closed-form. Third, vertical force constraints are added, after which capture regions are computed numerically using a bang-bang control law. The last capture region bridges the transition to the applied part of this work.

The second contribution is a control law on vertical acceleration, suitable for application on a humanoid robot using a momentum-based control framework. Push recovery is tested on NASA's Valkyrie humanoid robot while the robot is standing. In simulation, an increase in recoverable push of $9$\% can be observed when comparing to a controller that only uses \ac{CoP}, when pushing the back of the robot. On hardware, an average increase of $7$\% can be observed for this push direction using a load sensor. Additionally, tests are conducted on hardware on Boston Dynamics' Atlas using a medicine ball on a rope, but no improvement in recovery is observed. The control method for standing push recovery is also extended for use while the robot is walking. For Valkyrie in simulation, recovery improved the most compared with a predefined height approach for a push applied in the first part of the single support state for rear and frontal push directions. Additionally, hardware results on Atlas while walking are briefly presented.