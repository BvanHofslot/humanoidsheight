%
% Another appendix chapter
\chapter{Conclusion}\label{chap:conclusion}
In this work, novel capture regions for the \ac{VHIP} model are proposed, which addresses the \textit{theory} part of the research objective. Furthermore, control actions that use \ac{CoM} height variations for balance are presented and results are shown on hardware on humanoid robots, which addresses the \textit{application} part of the research objective. 

For the \ac{VHIP} model, capture regions are proposed in Chapter \ref{chap:regions} considering an unilateral contact constraint, after which height constraints and force constraints are added. It could be observed that the capture region becomes smaller after addition of constraints. Also, a comparison with the \ac{LIP} and \ac{LIP} plus flywheel capture regions is made, which gives a high level measure of the potential effects of \ac{CoM} height variation. The presented capture regions are derived under the assumption that kinematic limits and joint torque limits of the robot can be approximated with a constraint on minimum and maximum height and vertical acceleration respectively. As the vertical force constrained capture regions are computed numerically, there is experimented with another control law, an \ac{MPC} law, in Chapter \ref{chap:mpc}. Based on the shape of the control input of the \ac{MPC} however,, which is constrained to be a polynomial function, there is chosen to not use this control law in application.

Based on the control law used to compute the vertical force constrained capture regions, a control law is designed for implementation in a momentum-based whole-body control framework in Chapter \ref{chap:standing}. This bang-bang control law is activated when a predefined worst-case event is met. With this control law, push recovery tests are conducted on NASA's Valkyrie and Boston Dynamics' Atlas, while the robots are standing. 

The results for Valkyrie in simulation showed that push recovery improved $9$ \% after pushing the robot in the back and $4$ \% after pushing from the side when the bang-bang control law on vertical \ac{CoM} motion was used. Remarkably, push recovery was $7$ \% worse after a frontal push when enabling the bang-bang controller. The back push tests are also conducted on hardware, using a push stick and a load sensor, were an average of $7$ \% increase in maximum recoverable push was observed. Remarkably, the vertical force constrained capture position for the same height change and vertical acceleration differed approximately $4$ \% from the \ac{CP}, so differences are observed between the \ac{VHIP} model and the results on Valkyrie.

Additional hardware tests are also conducted on Atlas with a medicine ball on a rope. However, recovery did not improve noticeably. Different initial heights for the robot are tried, as well as a tuning of a joint gain, which had no effect. In simulation however, recovery did improve when enabling the bang-bang control action.

Similar to the bang-bang control action, three actions are proposed for the use in \ac{3D} in Chapter \ref{chap:walking}. Using two variables, a control action could be chosen heuristically based on an error. Compared to a constant height control approach, recovery improved the most when pushing the robot in the back or from the front in the first part of the swing phase. 

\section{Recommendations}
In the following sections, recommendations for future works are presented.
\subsection{Extension of Capture Regions}
The unilateral contact and height constrained capture regions give bounds on the capture region. However, they cannot directly be used in a control law. The vertical force constrained capture points can be used in control, but use numerical integration to find future state information. It would be interesting to explore closed-form solutions for a force constrained capture problem, without overly constraining the \ac{VHIP} like in \cite{pratt2007derivation} and \cite{koolen2016balance}. With a closed-form solution for example, the control law used in application in this thesis could be predictive, as a vertical force constrained capture position could be computed on every time instance.
\subsection{Improving Push Recovery Tests}
Balance control of the robots is tested in this work by testing push recovery. The load sensor measures push force accurately, but a person pushing cannot in general apply the same force in every test. With the tests with the medicine ball, the ball location could be put relatively precise. However, the push duration on the robot using these tests cannot be adjusted and is quite short, which resulted in high impacts on the robot. Furthermore, stretch in the rope and in the ball can change the \ac{CoM} height of the ball. For future push recovery tests, it could be interesting to use a device that can accurately apply force according to a desired profile over time.
\subsection{Balance Control on Humanoid Robots}
The work in this report covers merely a small part for balancing strategies of a humanoid robot. Under the assumption of the existence of a worst-case event, where additional control was needed, control actions are proposed. In this thesis, the control actions are solely chosen based on \ac{ICP} error and data from the \ac{ICP} reference trajectory.

Under the same assumption of worst-case events, it could be interesting to adjust the determination of these events on more complex information. Incorporating visual feedback of the robot could help determining such an event, and would lead potentially to more natural responses. For example, the robot could choose to use height variation, if it is facing a cliff. Also, the control law of \cite{caron2018balance} considers no further disturbance in the computation of a trajectory. In this thesis, the worst-case event is chosen conservatively, such that in some cases the use of height variation might be unnecessary. With for example visual feedback, the robot could estimate a duration of a disturbance.
\subsection{Analysis Techniques}
In this work, the data obtained from the robots are analyzed based on time-response and phase plots. It could be interesting to do extended analysis using traditional stability analysis techniques, as for example frequency-domain evaluation for time-varying systems.  Also, it could be interesting to analyze energy consumption on the robot, when using the different control strategies.