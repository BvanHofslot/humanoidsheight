%
% Another appendix chapter
\chapter{Conclusion}\label{chap:conclusion}
In this work, novel capture region for the \ac{VHIP} model are proposed. Furthermore, control actions that use \ac{CoM} height variations for balance are proposed. 

For the \ac{VHIP} model, capture regions are proposed considering an unilateral constraint, after which height constraints and force constraints are added. It could be observed that the capture region becomes smaller after addition of constraints. Also, a comparison with the \ac{LIP} and \ac{LIP} plus flywheel capture regions is made, which gives a high level measure of the potential effects of \ac{CoM} height variation. The presented capture regions are derived under the assumption that kinematic limits and joint torque limits of the robot can be approximated with a constraint on minimum and maximum height and vertical acceleration respectively.

Based on the control law used to compute the vertical force constrained capture regions, a control law is designed for implementation in a momentum-based whole-body control framework. This bang-bang control law is activated when a predefined worst-case event is met. With this control law, push recovery tests are conducted on NASA's Valkyrie and Boston Dynamics' Atlas, while the robots were standing. 

The results for Valkyrie in simulation showed that push recovery improved $9$ \% after pushing the robot in the back and $4$ \% after pushing from the side when the bang-bang control law on vertical \ac{CoM} motion was used. Remarkably, push recovery was $7$\% worse after a frontal push when enabling the bang-bang controller. The back push tests are also conducted on hardware, using a push stick and a load sensor, were an average of $7$ \% increase in maximum recoverable push was observed. Remarkably, the vertical force constrained capture position for the same height change and vertical acceleration differed approximately $4$ \% from the \ac{CP}, so differences are observed between the \ac{VHIP} model and the results on Valkyrie.

Additional hardware tests are also conducted on Atlas with a medicine ball on a rope. However, recovery did not improve noticeably. Different initial heights for the robot are tried, as well as a tuning on a joint gain, which had no effect. In simulation however, recovery did improve when enabling the bang-bang control action.

Similar to the bang-bang control action, three actions are proposed for the use in \ac{3D}. Using two variables, a control action could be chosen based on a error. Compared to a constant height control approach, recovery improved the most when pushing the robot in the back or from the front in the first part of the swing phase. 

\section{Recommendations}

 Couple timing adjustment with height control
 
 push recovery tests
 
 3d control
 
 analysis on robot
 
 balance strategies-> visual feedback
