%
% Another appendix chapter
\chapter{Conclusion}\label{chap:conclusion}
The objective of this work was to improve balance control of a humanoid robot using \ac{CoM} height variation. In this work, novel capture regions for the \ac{VHIP} model were proposed and compared with the \ac{LIP} model, which addressed the theoretical part of the research objective. Furthermore, control actions that use \ac{CoM} height variation for balance were presented and results were shown on hardware on humanoid robots in comparison with predefined \ac{CoM} height approaches. For Valkyrie, it was observed that balance was improved using \ac{CoM} height variation, which addressed the applied part of the research objective. 

For the \ac{VHIP} model, capture regions were proposed in Chapter \ref{chap:regions} considering a unilateral contact constraint, after which height constraints and force constraints were added. It was observed that the capture region becomes smaller after addition of constraints. Also, a comparison with the \ac{LIP} and \ac{LIP} plus flywheel capture regions was made, which gives a high-level measure of the potential effects of \ac{CoM} height variation. The presented capture regions were derived under the assumption that kinematic limits and joint torque limits of the robot can be approximated with a constraint on minimum and maximum height and vertical acceleration respectively. 

Because the vertical force constrained capture regions are computed numerically, there was experimented with another control law, a \ac{MPC} law, in Chapter \ref{chap:mpc}. Based on the shape of the control input of the \ac{MPC} however, which is constrained to be a polynomial function, there was chosen to not use this control law in applications in Chapter \ref{chap:standing} and Chapter \ref{chap:walking}.

Similar to the control law used to compute the vertical force constrained capture regions, a bang-bang control law was designed for implementation in a momentum-based whole-body control framework in Chapter \ref{chap:standing}. This control law is activated when a predefined threshold is met. With this control law, push recovery tests were conducted on NASA's Valkyrie and Boston Dynamics' Atlas, while the robots were standing. The results for Valkyrie in simulation showed that push recovery improved $9$\% when pushing the robot in the back and $4$\% when pushing from the side when the bang-bang control law on vertical \ac{CoM} motion was used. Remarkably, push recovery was $7$\% worse after a frontal push when enabling the bang-bang controller. The rear push tests were also conducted on hardware, using a push stick and a load sensor, where an average of $7$\% increase in maximum recoverable push was observed. The vertical force constrained capture position for the same height change and vertical acceleration differed approximately $4$\% from the \ac{CP}, so differences were observed between the \ac{VHIP} model and the results on Valkyrie. Additional hardware tests were conducted on Atlas using a medicine ball on a rope. However, recovery did not improve noticeably. Different initial heights for the robot were tried, as well as a tuning of a joint torque control gain, which had no noticeable effect. In simulation however, recovery did improve for Atlas when enabling the bang-bang control action.

Similar to the bang-bang control action, three actions were proposed for the use during walking in Chapter \ref{chap:walking}. Using the two presented variables, the alignment angle and the effective distance, a control action was chosen heuristically based on outputs of \ac{ICP} control. Compared to a constant height control approach, recovery improved the most when pushing the robot in the back or from the front in the first part of the swing phase. On hardware, evaluation of the proposed control law was difficult, because of the additional uncertainties compared to the standing tests. Therefore, only an example was shown for a control action on hardware on Atlas while the robot is walking.

\section{Recommendations}
The results presented in this work have demonstrated that \ac{CoM} height variations can improve balance control. There are however shortcomings, both on the theoretical as well as the applied side of the proposed research. In the following sections, recommendations for future work are presented. First, recommendations for extension of the proposed approaches are presented, after which a broader outlook on future work is briefly presented.
\subsection{Extending the Proposed Approach }
In this section, opportunities for improvement of the presented theory, tests and results are presented.
\subsubsection{Extension of Capture Regions}
The unilateral contact and height constrained capture regions give bounds on the capture region. However, these cannot directly be used in a control law, as impacts are considered in the computation. The vertical force constrained capture points can be used in control, but use numerical integration to find future state information. It would be interesting to explore closed-form solutions for a force constrained capture problem, without overly constraining the \ac{VHIP} like in \cite{pratt2007derivation} and \cite{koolen2016balance}. With a closed-form solution for example, the control law used in application in this thesis could be predictive, as a vertical force constrained capture position could be computed on every time instance.
\subsubsection{Improving Push Recovery Tests}
Balance control of the robots was tested in this work by testing push recovery. The push stick with load sensor \cite{iload} measures push force accurately, but a person pushing the robot is in general not able to apply a desired force precisely. With the tests with the medicine ball, the ball location could be put relatively precise. However, the push duration on the robot using these tests cannot be adjusted and is quite short, which resulted in high impacts on the robot. Furthermore, stretch in the rope and in the ball can change the \ac{CoM} height of the ball. Also, the transfer of the energy of the ball to the robot depends on the damping properties of the ball and the robot. For future push recovery tests, it could be interesting to use a device that can accurately apply force according to a desired profile over time.
\subsubsection{Improving State Estimation and Center of Pressure Control}
Applying the methods presented in this thesis requires good state estimation and control of the \ac{CoP}. In some experiments on Atlas it was found that performance did not improve when applying the presented control law, even though performance could improve according to the \ac{VHIP} model and the obtained simulation results. This lack of performance could be related to a \ac{CoP} error, which could be caused by the additional movement of the robot when the presented method was used. By improving state estimation and \ac{CoP} control, the theoretically predicted improvements could potentially be better achieved in practice.
\subsubsection{Standing Tests for Lowering Center of Mass Height}
The standing push recovery tests presented in this work all use an increase in \ac{CoM} height for balance control, as the initial \ac{CoM} position of the robot was inside the support polygon at the moment the push was applied. With the walking tests, the \ac{CoM} height was lowered. However, the walking tests were difficult to test on hardware, because of the increased number of uncertainties. It could be interesting to find a test setup, where the robot should lower the \ac{CoM} height to balance to a standing configuration. The robot can be given an initial velocity when the \ac{CoM} is outside the support polygon, that lowering the \ac{CoM} height would be needed to balance. This test setup would be comparable with a human landing after a long jump.
\subsubsection{Analysis of Results}
In this work, the data obtained from the robots was predominantly analyzed based on time response and phase plots. It could be interesting to perform additional analysis, as for example analyzing the energy consumption on the robot when using the different control strategies. The energy consumption could be determined by measuring the electric current and voltage going to the robot. Energy consumption can, for example, be an additional decision variable for choosing between balancing strategies.

\subsection{Outlook}
This work contributes merely a small part to balancing strategies for humanoid robots. For the future, it could be interesting to analyze \ac{CoM} height variations in legged systems further. It would be interesting to investigate when humans use height variation, and why the strategy is chosen instead of the `hip strategy' in such scenarios for example. Furthermore, it would be interesting to see different balancing strategies combined with height variation on humanoid robots. The decision making in balancing strategies, under the constraints of kinematic limits, force limits, disturbances and terrain remains a broad research area to explore.