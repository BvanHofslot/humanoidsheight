%
% Introduction
\chapter{Introduction} \label{chap::intro}
\section{Motivation}

Different reasons to research varying \ac{CoM} height: 
\begin{itemize}
	\item Improve behavior over rough-terrain
	\item Minimize energy consumption or mimic natural behavior
	\item Analyse the effects of height variation
	\item Extend control authority by using height variations
\end{itemize}
A common approach in humanoid robotics is to define the system as a \ac{LIP} with finite-sized foot and a mass with inertia. The ankle and the angular momentum about the \ac{CoM} can be used as control inputs to generate horizontal forces on the \ac{CoM}, often called "ankle" and "hip" strategies. Allowing for the vertical component of the \ac{GRF} to vary, more horizontal force can be generated as well.
\section{Research Objective}
In this thesis is focussed on the last two goals. Analyse the effects of height variation. Extend the control authority by using height variations. 

\section{Contributions}
\begin{itemize}
	\item Theoretic Limits on Capture
	\item Orbital Energy MPC
	\item Approaches for application on real robot
\end{itemize}
\section{Thesis Outline}


