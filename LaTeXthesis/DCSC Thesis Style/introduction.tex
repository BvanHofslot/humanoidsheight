%
% Introduction
\chapter{Introduction} \label{chap::intro}
Humans are often exposed to tasks that could harm them. That is for example the case in disaster response: entering a radioactive environment to help people after a nuclear disaster, or entering a building on fire to save a life. Next to the danger of harm, factors like time and financials can also play a role. Take for example an exploration space mission to Mars, where a human committing to such a task is in risk of harm, but also costs a lot of time and money. In all these situations, the physical versatility of the human, such as grasping with hands and walking over uneven terrain, is needed. Substitution of a human-like machine in such scenarios, which has the same benefits as a human, could be very beneficial. 

Replacing the physical human with a machine is not a new topic. There already exist texts dating before Christ, that describe a human-like machine \cite{behnke2008humanoid}. As technology became more advanced, the topic of humanoid robotics became more popular in the last decades. However, a humanoid robot that has the same physical capabilities as an average human being, still not yet exists. One of those physical capabilities is keeping balance. Humans are capable of not falling over in various terrains, configurations and subject to disturbances, while robots are frequently not. 

Commonly used balancing strategies for humanoid robots are control of the \ac{CoP}, `ankle' strategies, and to a lesser extent, change of body angular momentum, for example: `hip' strategies. If additional steps are included in the problem, also modification of the step time and footstep location are considered as balancing strategies. An important assumption behind these strategies is often that the \ac{CoM} height of the robot is predefined and not to be used in balancing tasks. Reasons for this include that \textit{dynamic planning} is often conducted with a \ac{LIP} model \cite{kajita20013d}. An important advantage of the use of this model is that, by its linearity, closed-form solutions exist for the solution of the dynamics: the \ac{LIP} \textit{orbital energy} \cite{kajita1992dynamic}. The \ac{CP} \cite{pratt2006capture} and \ac{ICP} \cite{koolen2012capturability}, the `extrapolated center of mass' \cite{hof2008extrapolated} and the \ac{DCM} \cite{takenaka2009real} all rely on the solution of the \ac{LIP} dynamics. For \textit{kinematic feasibility} or more human-like motion, height trajectories that differ from the \ac{LIP} height  can be generated, but are disturbances on the dynamic model. These disturbances have to be controlled with balance control strategies. Recently, the use of vertical \ac{CoM} motion in balance control appeared as a relatively new research area to explore.

\section{Related Work}
Though, the use of a variable height model in dynamic planning is a relatively longer existing research field. In \cite{pratt2007derivation}, Pratt \& Drakunov derived a \ac{2D} solution for the \textit{orbital energy} for a \ac{VHIP}. By constraining the height trajectory to be a function of the horizontal position, an analytic solution for energy became available. In \cite{englsberger2013three}, Englsberger, Ott \& Albu-Sch{\"a}ffer proposed a method using the \ac{3D} \ac{DCM} to generate a dynamic plan. This \ac{DCM} can be seen as a \ac{3D} version of the \ac{ICP} \cite{koolen2012capturability}, which allowed to linearly interpolate differences in height between \ac{DCM} knot-points in the generation of a dynamic plan.  This \ac{DCM} is extended to a time varying version by Hopkins, Hong \& Leonessa in \cite{hopkins2014humanoid}. The \ac{3D} \ac{DCM} was made time varying, such that any trajectory shape between knot-points could be used in the generation of a dynamic plan. However, an analytic solution was not available anymore and dynamic planning had to conducted numerically. 

Recently, the use of vertical \ac{CoM} motion came in sight for the particular goal to improve balance control. Koolen, Posa \& Tedrake developed an analytic \ac{MPC} law in \cite{koolen2016balance}, using the \ac{VHIP} orbital energy derived in \cite{pratt2007derivation}. A study was conducted on the \textit{regions of attraction} of the controller: the domain of initial conditions of the pendulum-based model after which stability was still possible, given the model. These regions are comparable with the \textit{capture regions} proposed by Pratt, Carff, Drakunov \& Goswami \cite{pratt2006capture}, only the capture regions relied on a \ac{LIP} model. The controller was able to compute \ac{2D} closed-form solutions to the trajectory to come to a stop, but without taking kinematic or force constraints into account. In \cite{gao2017increase}, Gao, Jia \& Fu presented strategies in \ac{2D} using vertical motion. Lowering the \ac{CoM} in the current step, and applying more force and raising the \ac{CoM} in the next step, is an example of a strategy presented in the work. The dynamic model considered in the strategies is a \ac{LIP}, but with an adjusted natural frequency by a constant vertical acceleration. In \cite{caron2018balance}, the same problem was approached in \ac{3D} by Caron \& Mallein. The \ac{MPC} law was able to compute \textit{capture trajectories} for different foot orientations and initial conditions in three dimensions online, using also a \ac{VHIP} model.

\section{Research Objective}
The objective of this work is to improve balance of a humanoid robot using vertical \ac{CoM} motion. To measure \textit{improving balance}, a high-level differentiation can be made in:
\begin{itemize}
	\item Analysis of a \ac{VHIP} and comparison with the \ac{LIP}: \textit{theory}.
	\item Analysis of improvements observed in simulation and on hardware of a robot, using a common control framework to transcribe pendulum-based control commands to the joints of the robot: \textit{application}.
\end{itemize}

\section{Contributions}
The contributions of this thesis are:
\begin{itemize}
	\item Theoretic capture regions: different bounds on  the capture region are derived for a \ac{VHIP} model. 
      \item A constrained orbital energy \ac{MPC}: with the \ac{2D} method of \cite{koolen2016balance} a \ac{MPC} law is introduced that takes kinematic constraints into account.
	\item A method to use vertical \ac{CoM} motion in an applied setting in \ac{2D} and \ac{3D}, and application of the method on NASA's Valkyrie humanoid robot.
\end{itemize}
\section{Thesis Outline}
The thesis is structured as follows. In Chapter \ref{chap:background}, background is given on walking in humanoid robotics, related works to vertical \ac{CoM} motion and humanoid robotics at \ac{IHMC}. In Chapter \ref{chap:regions}, theoretic capture regions are proposed. In Chapter \ref{chap:mpc}, an \ac{2D} \ac{MPC} law is derived. In Chapter \ref{chap:standing}, a controller for \ac{2D} is presented and push recovery is tested on NASA's Valkyrie during stance. In Chapter \ref{chap:walking}, the \ac{2D} controller is extended to \ac{3D} and push recovery is tested on Valkyrie during walking in simulation. The final chapter, Chapter \ref{chap:conclusion}, presents the conclusion and recommendations for future work.

\subs{d}{Desired}
\subs{r}{Reference}
\subs{xy}{Horizontal part selected of 3D vector}
\subs{z}{Vertical part selected of vector}