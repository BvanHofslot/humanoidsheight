%
% Introduction
\chapter{Introduction} \label{chap::intro}

This is a \LaTeX\ thesis and this is Chapter\ \ref{chap::intro}.

\section{About \texorpdfstring{\LaTeX}{LaTeX}}

\LaTeX\ is a document preparation system for the \TeX\ typesetting program. It offers programmable desktop publishing features and extensive facilities for automating most aspects of typesetting and desktop publishing, including numbering and cross-referencing, tables and figures, page layout, bibliographies, and much more.

\LaTeX\ was originally written in 1984 by Leslie Lamport and has become the dominant method for using \TeX; few people write in plain \TeX\ anymore. The current version is \LaTeXe.

If you want to know more about \LaTeX\ you better read \cite{texbook}.\index{LaTeX}


\section{About Acronyms}

This section contains an acronym of the \ac{DCSC}. The \ac{DCSC} is our department within the faculty of \ac{3mE} at \ac{TU}. \index{acronym}

Acronyms are automatically listed in the Glossary in the back of this thesis. You have to define acronyms in \texttt{glossary.tex} using \verb"\acro{ACRONYM}{Full text}". You print an acronym by using the command \verb"\ac{...}". You can always force a full, long or short printout by using \verb"\acf{...}", \verb"\acl{...}" or \verb"\acs{...}" respectively.

\begin{itemize}
    \item \verb"\acf{DCSC}": \acf{DCSC};
    \item \verb"\acl{DCSC}": \acl{DCSC};
    \item \verb"\acs{DCSC}": \acs{DCSC}.
\end{itemize}

\section{About the Nomenclature}

When you use symbols in your thesis -- as you probably will -- you can put them into the nomenclature listing (List of Symbols) at the back of your thesis. \tabref{tab:nomencl} shows the \LaTeX\ commands you need.\index{nomenclature}

\begin{table}%
    \centering
    \caption{Nomenclature codes}
    \label{tab:nomencl}
    \begin{tabular}{llcl}
        \toprule
        Code & Usage & Example\\
        \midrule
        \verb"\gsymb{}" & Greek symbols & \gsymb{$\gamma$}{Path Angle}\\
        \verb"\lsymb{}" & Letter symbols & \lsymb{$H(s)$}{Transfer function}\\
        \verb"\supers{}" & Superscript symbols & \supers{max}{Maximum} &\emph{only printed in the List of Symbols} \\
        \verb"\subs{}" & Subscript symbols & \subs{min}{Minimum} &\emph{only printed in the List of Symbols}\\
        \verb"\others{}" & Other symbols & \others{[kts]}{Knots} \others{$^{\circ}$, [deg]}{Degrees} &\emph{only printed in the List of Symbols}\\
        \bottomrule
    \end{tabular}
\end{table}

\section{About {\textbackslash}(re)newcommand}
As you will (soon) know the \LaTeX\ system makes use of commands in
the form of \verb"\command". This can be used to make your life
easier, since you can also define these commands yourself. Suppose
that you often use an expression $e^{it}$. This would
normally be written as \verb"$e^{it}$", or if already in math mode
as \verb"e^{it}". Now you can define a command
\verb"\eit" as follows\\
\verb"\newcommand{\eit}{e^{it}}"\\
This definition has to be placed in the so-called preamble, i.e.
before the declaration\\ \verb"\begin{document}".\\ Now you can use
this
command in your text, so \verb"\eit" replaces \verb"e^{it}".\\
Be aware that many commands are already in use by various packages.
If you define an already existing command this will result in an
error message. The best way to deal with this is to make sure your
own command is unique, for instance by defining it as\\
\verb"\newcommand{\MYeit}{e^{it}}".\\
An alternative is to redefine the existing command by\\
\verb"\renewcommand{\eit}{e^{it}}",\\
but this is in general considered a tedious practice and should be
avoided.
See the \LaTeX\ documentation for more details and possibilities. You can also use arguments.\\
%
Don't underestimate the power of this feature. Suppose that you
frequently use the expectation operator $E(x)$ in your text which is
created with \verb"E(x)" (in math mode). Now your supervisor decides
that he would prefer to see this as $\bf{E}(x)$. If you haven't
defined your own command, you will have to go through the complete
text, changing every instance. If you would have thought about it
and
would have defined originally\\
\verb"\newcommand{\MyE{1}}{E(#1)}",\\
using this in your text as
\verb"\MyE{x}", \verb"\MyE{y}" etc.,\\
then you can change this simply by adapting the definition to\\
\verb"\newcommand{\MyE{1}}{\bf{E}(#1)}".
