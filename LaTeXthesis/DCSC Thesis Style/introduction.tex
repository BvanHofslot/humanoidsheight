%
% Introduction
\chapter{Introduction} \label{chap::intro}
\section{Motivation}
A common approach in humanoid robotics is to define the system as a \ac{LIP} with finite-sized foot and a mass with inertia. The ankle and the angular momentum about the \ac{CoM} can be used as control inputs to generate horizontal forces on the \ac{CoM}, often called "ankle" and "hip" strategies.  Next to this, the step timing and footstep location can be adjusted
\begin{itemize}
	\item \ac{CoP}
	\item Angular momentum
	\item Change step time
	\item Change step location
\end{itemize}

Allowing for the vertical component of the \ac{GRF} in the model to vary, and thus the height to vary, the horizontal components of the \ac{GRF} change as well, compared to the \ac{LIP} model.\\
Different reasons to research varying \ac{CoM} height: 
\begin{itemize}
	\item Improve behavior over rough-terrain
	\item Minimize energy consumption or mimic natural behavior
	\item Analyse the effects of height variation
	\item Extend control authority in horizontal direction by using height variations
\end{itemize}

[SOMEWHERE: DYNAMIC AND KINEMATIC FEASIBILITY + torque limits?]\\
HYPOTHESE?
\section{Research Objective}
In this thesis is focussed on the last two goals. Analyse the effects of height variation. Extend the control authority by using height variations. 

\section{Contributions}
\begin{itemize}
	\item Theoretic limits on capture: different limits on capture are derived, considering height variation.
	\item Orbital energy \ac{MPC}: with the \ac{2D} method of \cite{koolen2016balance} a constrained \ac{MPC} is introduced.
	\item Approach for application on real robot: approaches are derived to implement in the control framework at IHMC.
\end{itemize}
\section{Thesis Outline}


\subs{d}{Desired}
\subs{r}{Reference}
\subs{xy}{Horizontal part selected of 3D vector}
\subs{z}{Vertical part selected of vector}