%
% Another appendix chapter
\chapter{Experimental Setup}

%standing setup
\section{Standing Push Recovery}
\subsubsection{Simulation}
360 push.
\subsubsection{Hardware}
\begin{itemize}
	\item Increase $\ddzdmax$ until best found
	\item Front and side push
	\item Compare with normal config
	\item Compare with QP-based
	\item Compare with ang momentum
\end{itemize}
%preliminary observations
\subsubsection{Preliminary Observations}
\begin{itemize}
	\item Changing weight in momentum optimization settings already trigger height variation.
\end{itemize}

%Walking setup
\section{Walking Push Revovery}
8 push directions.\\
Atlas or Valkyrie walking over a terrain with limited foot placement options, so step adjustment is not possible (tiled environment). Push recovery to test disturbance rejection. See \tabref{tab:stepping} for the used parameters. The dynamic plan, footstep plan and \ac{ICP} plan, are left untouched.
\begin{table}[ht]
\caption{Stepping Parameters} % title of Table
\centering % used for centering table
\begin{tabular}{c c c } % centered columns (4 columns)
\hline\hline %inserts double horizontal lines
Parameter & Value & Unit \\
%heading
\hline % inserts single horizontal line
Step Legth & 0.5 &  [m]\\
Step Width & 0.25 & [m]\\
\acs{SS} Time & 0.6 & [s]\\
\acs{DS} Time & 0.15 & [s]\\
%[1ex] % [1ex] adds vertical space
\hline %inserts single line
\end{tabular}
\label{tab:stepping} % is used to refer this table in the text
\end{table}
%Preliminary observations
\section{Preliminary Observations}
\begin{itemize}
	\item Direction \ac{ICP} error stays often the same during swing
	\item If the \ac{ICP} error is directed in the sagittal plane, CMP often remains somewhat in the same location
	\item if the \ac{ICP} error is more directed  in the coronal plane, CMP slides from back to forth foot. 
	\item The configuration and velocity at near touch down is important for: swing leg touch down and swing leg collapse in \ac{DS}
\end{itemize}

